\documentclass{article}
\usepackage{graphicx} % Required for inserting images
\usepackage{amsmath}
\usepackage{amssymb}
\usepackage{indentfirst}

\title{Daily population-density-weighted weather raster per year per district}
\author{Tuyen Huynh}
\date{December 2024}

\begin{document}

\maketitle

% \section{Introduction}

\section{Definition and formulas}

\subsection{Closed polygon}

A closed polygon is a sequence of ordered points (vertices) which forms a two-dimensional geometric figure. The geometric shape of a district is considered a closed polygon.

A closed polygon $L$ is defined as:
\begin{equation*}
    L = \{(x_1, y_1), (x_2, y_2), ..., (x_n, y_n), (x_1, y_1)\}
\end{equation*}

where:
\begin{itemize}
    \item $(x_i, y_i)$ are coordinates of the $i$-th vertex of the polygon
    \item $n$ is the number of vertices
    \item The first and last points are the same, ensuring the polygon is closed
\end{itemize}

\subsubsection{Interior of a polygon}

Interior of a polygon is the of points that are inside the polygon but not on its edges or vertices. Denoted as:

\begin{equation} \label{eq:interior_poly}
    \operatorname{Int}(L) = \{x \in \mathbb{R}^2 \mid x \text{ enclosed by edges of } L\}
\end{equation}


\subsubsection{Boundary of a polygon}

Boundary of a polygon is the set of points that form its edges and vertices. Denoted as:

\begin{equation} \label{eq:boundary_poly}
    \operatorname{Bnd}(L) = \bigcup_{i=1}^{n}{e_i} \in \mathbb{R}^2
\end{equation}

where ${e_i}$ are the edges of the polygon $L$

\subsubsection{Closure of a polygon}

The closure of a polygon $L$ is the union of its interior and boundary, given (\ref{eq:interior_poly}) and (\ref{eq:boundary_poly}) we can denote this as:

\begin{equation}
    \operatorname{Closure}(L) = \operatorname{Int}(L) \cup \operatorname{Bnd}(L)
\end{equation}

A closure represent the complete polygon, i.e. all the points that makes up the polygon.


\subsection{What is a raster}

A raster $R$ is defined as a matrix of raster cells $r$ with dimensions $(n_\text{rows}, n_\text{cols})$, a spatial resolution of $(\Delta{x},\Delta{y})$, and an origin point of $(x_\text{min}, y_\text{max})$. $n_\text{rows}$ is a number of rows in the matrix, and $n_\text{cols}$ is a number of cols in the matrix. It is assumed that $\Delta{x} = \Delta{y}$. $R$ is denoted as:

\begin{equation} \label{eq:raster_def}
    R = \{r_{i,j} \mid i \in [1, n_\text{rows}], j \in [1, n_{\text(cols)}]\}
\end{equation}

% or $R$ can also be denoted as:

% \begin{equation} \label{eq:raster_def_vect}
%     R = \{r_{k} \mid k \in [1, n_\text{cells}]\}
% \end{equation}

% where $n_\text{cells} = n_\text{rows} \cdot n_\text{cols}$ is the number of cells in a raster.

Each raster cell also store a value, e.g. for a weather raster, a cell would store values of temperature or precipitation; for a population density raster, each cell would store a value of population density. Performing summation over a set of raster cells will imply performing summation over the values of the raster cells, denoted as $\sum(R)$.

\subsection{Geometry of a raster}

A raster can be converted into a geometry shape by converting each of its cells into a square. The conversion for each cell is defined as:

\begin{equation} \label{eq:rast2geom}
    \begin{split}
        \operatorname{Geom}(r_{i,j}) = \{
            &(x_{\text{min}}+(j-1)\cdot\Delta{x}, y_{\text{max}}-(i-1)\cdot\Delta{y}),\\
            &(x_{\text{min}}+j\cdot\Delta{x}, y_{\text{max}}-(i-1)\cdot\Delta{y})\\
            &(x_{\text{min}}+j\cdot\Delta{x}, y_{\text{max}}-i\cdot\Delta{y})\\
            &(x_{\text{min}}+(j-1)\cdot\Delta{x}, y_{\text{max}}-i\cdot\Delta{y})\\
            &(x_{\text{min}}+(j-1)\cdot\Delta{x}, y_{\text{max}}-(i-1)\cdot\Delta{y})
        \}
    \end{split}
\end{equation}

where:
\begin{itemize}
    \item \textbf{Top-left}: $(x_{\text{min}}+(j-1)\cdot\Delta{x}, y_{\text{max}}-(i-1)\cdot\Delta{y})$
    \item \textbf{Top-right}: $(x_{\text{min}}+j\cdot\Delta{x}, y_{\text{max}}-(i-1)\cdot\Delta{y})$
    \item \textbf{Bottom-right}:$(x_{\text{min}}+j\cdot\Delta{x}, y_{\text{max}}-i\cdot\Delta{y})$
    \item \textbf{Bottom-left}: $(x_{\text{min}}+(j-1)\cdot\Delta{x}, y_{\text{max}}-i\cdot\Delta{y})$
\end{itemize}

Given (\ref{eq:raster_def}) and (\ref{eq:rast2geom}), a raster $R$ can then be also converted into a geometry $R^*$ defined as a collection of polygons:

\begin{equation} \label{eq:rast_geom_set}
    R^* = \{ \operatorname{Geom}(r_{i,j}) : r_{i,j} \in R \}
\end{equation}

% \subsection{Intersection}
% Defining how a polygon $L_1$ intersects with polygon $L_2$, i.e. 2 polygons sharing any space together:

% \begin{equation}
%     I = \{r_{i,j} \in R \mid \operatorname{Geom}(r_{i,j}) \cap S \neq \emptyset\}
% \end{equation}

\subsection{Coverage}

Defining a filtering function where given a set of polygons $L_2$, select only polygons that are covered (contained) by a polygon $L_1$:

\begin{equation}
    \operatorname{Cover}(L_1, L_2) = \{
        \operatorname{Closure}(l) \subset \operatorname{Closure}(L_1) \mid l \in L_2
    \}
\end{equation}

Hence, given a set of polygons $R^*$ such as ones from (\ref{eq:rast_geom_set}), we can filter polygons from $R^*$ that are covered by a polygon $L$ with:
\begin{equation}
    L' = \{ \operatorname{Cover}(L, r^*) : r^* \in R^* \}
\end{equation}


\subsection{Cropping}

Denote a function that crops a raster $R$ using a polygon $L$, resulting in a set of raster cells that intersect with, and inside, the polygon:

\begin{equation}
    \operatorname{Crop}(R, L) = \{
        \operatorname{Inter}(\operatorname{Geom}(r), L)
        \cup
        \operatorname{Cover}(L, r)
        :r \in R
    \}
\end{equation}

\section{Aggregation}

For each district $d$, day $t$, and weather variable $W$ from weather raster $R$:

\begin{equation}
    W_d(t) = \sum_{p=1}^{N_{\operatorname{Inter}(R, L_d)}}\left( 
        W_p(t) \cdot \frac{\sum(S'_p)}{}
    \right)
\end{equation}

where

\begin{equation*}
    S'_p = \{ \operatorname{Cover}(\operatorname{Geom}(R_p), s^*_{m,n}) : s^*_{m,n} \in S^* \},
\end{equation*}
\begin{equation*}
    S^* = \{\operatorname{Geom}(s) : s \in \operatorname{Crop}(S,L_d)\}
\end{equation*}

\begin{itemize}
    \item $S'_p$ is the set of population density raster cells covered by $R_p$
    \item $S^*$ is the set of polygons of raster cells from a population density raster $S$  (defined as (\ref{eq:rast_geom_set})), cropped by district geometry $L_d$
    \item $R_p$ is a raster cell from $R$ and $p \in [1, n_\text{cells}]$

\end{itemize}

% \begin{itemize}
    % \item $P$: Population density raster 
    % \item $N_{\operatorname{Inter(R, G_d)}}$: Number of weather pixels intersecting with district geometry $L_d$
    % \item $P_i$: Population density value for weather pixel $i$ (calculated as the sum of population density pixel values covered by district geometry $d$).
    
% \end{itemize}


\end{document}
